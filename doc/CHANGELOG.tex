\documentclass{article}

\usepackage{polski}
\usepackage[utf8]{inputenc}
\usepackage{amsmath}
\usepackage{hyperref}

\title{Sprawozdanie z symulacji}
\author{Robert Kraut}
\begin{document}
    \maketitle
    \section{Informacje wstępne}
    Zostały zebrane dane o serwerze ze strony \href{https://projectreactor.io/docs/netty/release/reference/index.html}{reactor-netty}. \\
    Maksymalna długość kolejki oczekujących żądań wynosi 1000, maksymalna ilość połączeń 500 a 45s\newline
    to maksymalny czas na uzyskanie połączenia.
    \section{Model}
    Ze względu na ograniczony czas pomijam specyfikację TCP/IP i skupiam się tylko na warstwie aplikacji \newline
    modelu ISO/OSI i protokole HTTP w wersji 1.0.\newline
    Rozważanymi parametrami będą:
    - dla wejściowych:
        - ilość żądań na sekundę
    - wyjściowe:
        - czas jaki upłynął od momentu zaakceptowania żądania do splasowania odpowiedzi w buforze
    - stałe:
        - czas wykonania określonego zadania
        - długość kolejki
        - czas umeszczenia danych w buforze
    \section{Etap 1 - Przeprowadzenie rzeczywistej symulacji o małym obciążeniu}


\end{document}
